%!TEX TS-program = xelatex
\documentclass[print]{friggeri-cv}
\addbibresource{bibliography.bib}

\begin{document}
\header{Per Thomas }{Lundal}
       {Take it apart - Discover it's secrets}


% In the aside, each new line forces a line break
\begin{aside}
    \section{About}
        Birkebeinervegen 12
        7037 Trondheim
        Norway
        ~
        +47 930 51 129
        ~
        \href{mailto:perthomas@gmail.com}{perthomas@gmail.com}
        \href{http://github.com/lundal}{http://github.com/lundal}
        \href{http://lund.al}{http://lund.al}
    \section{Languages}
        Native Norwegian
        Fluent English
        ~
        Java, C\#, C
        Python
        Assembly
        VHDL
\end{aside}

\section{Interests}

Microcontrollers, computer architecture and design, graphics, algorithms, FPGA

\section{Education}

\begin{entrylist}
    \entry
        {Aug 10 – Now}
        {Master of Science}
        {Norwegian University of Science and Technology}
        {Majoring in Computer Science\\
        Specialization in Complex Systems}
    \entry
        {Aug 07 – Jun 10}
        {High School}
        {Trondheim Katedralskole}
        {General Studies\\
        Specialization in Science}
    \entry
        {Aug 04 – Jun 07}
        {Secondary School}
        {Sunnland Skole}
        {}
    \entry
        {Aug 97 – Jun 04}
        {Primary School}
        {Nardo Skole}
        {}
\end{entrylist}

\section{Experience}

\begin{entrylist}
    \entry
        {Apr 05 – Now}
        {\href{http://biltema.no/}{Biltema}}
        {Trondheim, Norway}
        {Part Time Store Assistant\\
        Customer Service, Sales, Unboxing}
    \entry
        {Jun 14 – Aug 14}
        {\href{http://mesan.no/}{Mesan}}
        {Oslo, Norway}
        {Summer Intern\\
        IT Consulting}
\end{entrylist}

\section{Projects}

\begin{entrylist}
    \entry
        {Aug 13 – Dec 13}
        {\href{http://barricel.li/}{Barricelli - A high-performance genetic algorithms computer}}
        {}
        {A specialized genetic algorithms computer designed and built by 10 students at the Norwegian University of Science and Technology. It is designed to quickly find good approximations to hard problems using a custom-made hardware-accelerated generic genetic algorithm solver.}
    \entry
        {Jan 13 – Jun 13}
        {\href{https://github.com/lundal/avr32-project-2013}{AVR32 Microcontroller Project}}
        {}
        {A group project in for the STK1000 development board featuring an AVR32 microcontroller, in three parts:\\
        - A LED and button demo written in assembly\\
        - A MIDI player written in C\\
        - A game written in C for Linux with custom drivers for LEDs and buttons}
\end{entrylist}

\end{document}
