%!TEX TS-program = xelatex
\documentclass[print]{friggeri-cv}
\addbibresource{bibliography.bib}

\begin{document}
\header{Per Thomas }{Lundal}
       {Take it apart - Discover it's secrets}


% In the aside, each new line forces a line break
\begin{aside}
    \section{Contact}
        Birkebeinervegen 12
        7037 Trondheim
        Norway
        ~
        \href{mailto:perthomas@gmail.com}{perthomas@gmail.com}
        +47 930 51 129
        ~
        \href{http://github.com/lundal}{http://github.com/lundal}
        \href{http://lund.al}{http://lund.al}
    \section{Languages}
        Native Norwegian
        Fluent English
        Basic Spanish
        ~
        Java, C\#, C
        Python
        Assembly
        VHDL
\end{aside}

\section{Interests}

Microcontrollers,
Computer Architecture and Design,
FPGA,
Evolutionary Hardware,
Algorithms,
Graphics,
Artificial Intelligence,
Security,
Performance,
Science,
3D Printing,
Games

\section{Education}

\begin{entrylist}
    \entryx
        {\href{http://ntnu.no/}{Norwegian University of Science and Technology}}
        {Master of Science in Computer Science}
        {Trondheim, Norway}
        {Aug 10 – Jun 15 (Expected)}
        {- Specialization in computer architecture and design\\
         - Additional courses in graphics and artificial intelligence\\
         - Average grade of B}
\end{entrylist}

\section{Experience}

\begin{entrylist}
    \entryx
        {\href{http://biltema.no/}{Biltema}}
        {Store Assistant, Part Time}
        {Trondheim, Norway}
        {Apr 05 – Present}
        {- Customer service, sales and unboxing}
    \entryspace
    \entryx
        {\href{http://mesan.no/}{Mesan}}
        {IT Consultant, Summer Internship}
        {Oslo, Norway}
        {Jun 14 – Aug 14}
        {- Creation of a demo-application for Microsoft's health-initiative\\
         - Target platforms were Windows 8.1 and Windows Phone 8.1\\
         - Technologies used include Azure, REST, XAML and SQLite}
    \entryspace
    \entryx
        {\href{http://ntnu.no/}{Norwegian University of Science and Technology}}
        {Student Assistant, Part Time}
        {Trondheim, Norway}
        {Aug 13 – Dec 13}
        {- TDT4165 Programming Languages}
\end{entrylist}

\section{Projects}

\begin{entrylist}
    \entryy
        {\href{https://github.com/lundal/carp}{Cellular Automata Research Platform}}
        {Aug 14 – Present}
        {CARP is a platform for evolvable hardware research using cellular automata.
         It consists of a configurable cellular automata along with dedicated development and fitness modules implemented on an FPGA.
         My contribution is a new PCIe communication module and a complete rewrite of old code to improve maintainability, add new features and squish bugs.}
    \entryspace
    \entryy
        %{\href{http://barricel.li/}{Barricelli - A high-performance genetic algorithms computer}}
        {\href{http://lund.al/barricel.li/}{Barricelli - A high-performance genetic algorithms computer}}
        {Aug 13 – Dec 13}
        {Barricelli is a specialized genetic algorithms computer designed and built by 10 students at the Norwegian University of Science and Technology.
         It is designed to quickly find good approximations to hard problems using a custom-made hardware-accelerated generic genetic algorithm solver.
         My contribution was a large part of the FPGA design and implementation.}
    \entryspace
    \entryy
        {\href{https://github.com/lundal/avr32-project-2013}{AVR32 Microcontroller Project}}
        {Jan 13 – Jun 13}
        {A three-man project for the STK1000 development board which features an AVR32 microcontroller, in three parts:\\
        - A LED and button demo written in assembly\\
        - A MIDI player written in C\\
        - A game (including engine) written in C for Linux with custom drivers for LEDs and buttons}
\end{entrylist}

\end{document}
